\section{MDPs lösen: Zustands- und Aktionswerte}
\label{section:mdp_fundamentals}
Bei der Einführung der Notation wird sich an das Lehrbuch \cite{Sutton1998} gehalten.
Gegeben sei ein Markov-Entscheidungsprozess $(\mathcal{S}, \mathcal{A}, p)$ (kurz MDP).
Dabei ist $\mathcal{S}$ der Zustands- und $\mathcal{A}$ der Aktionsraum.
Weiterhin ist für einen Zustand $s\in \mathcal{S}$ und eine Aktion $a \in \mathcal{A}$ die Wahrscheinlichkeitsdichte, in einen Zustand $s' \in \mathcal{S}$ überzugehen und dabei die Belohnung (engl. Reward) $r \in \mathbb{R}$ zu erhalten, als $p_{s,a}(s', r)$ gegeben.
Zentraler Begriff beim Lösen eines MDPs ist die sogenannte Policyfunktion.
Sie kann deterministisch sein und als $\pi_{det}: \mathcal{S} \rightarrow \mathcal{A}$ formalisiert werden.
Weiterhin können Policies stochastisch sein.
Die Policyfunktion ist dann eine Zähl- oder Wahrscheinlichkeitsdichtefunktion der Form $\pi_{stoch}: \mathcal{S}\times\mathcal{A}\rightarrow [0, \infty)$.
Mit der Policy, den Zustandsübergängen, Belohnungen und dem Startzustand $S_0$ ergibt sich dann eine Episode als $\tau = \left(S_0, A_0, R_0, S_1, A_1, R_1, \dots, S_{n_{ep} - 1}, A_{n_{ep} - 1}, R_{n_{ep} - 1}\right)$.
Dabei sind für jedes $i \in \left\lbrace0, \dots, n_{ep} - 1\right\rbrace$ stets $S_i$, $A_i$, $R_i$ Zufallsvariablen, die für den $i$-ten Schritt Zustand, Aktion und Belohnung angeben.
Mit einem optionalen Diskontierungsfaktor $\gamma \in [0, 1]$ werden die addierten Belohnungen
\begin{equation}
	G_t := \sum_{i=t}^{n_{ep} - 1} \gamma^{i - t} \cdot R_i
\end{equation}
als Return (von Schritt $t$ aus) bezeichnet.
In einer infinite-horizon-Formulierung muss für eine endliche Summe notwendigerweise $\gamma < 1$ gewählt werden.
Ziel ist, den Return der ganzen Episode, also $G_0$ zu maximieren.

Für einen Zustand $s \in \mathcal{S}$ unter einer Policy $\pi$ gibt die Funktion
\begin{equation}
	V_\pi(s) := \mathbb{E}_\pi\left[G_t | S_t = s\right]
\end{equation}
den sogenannten Zustandswert an.
Er drückt die (diskontierte) Summe aller Rewards aus, die nach dem Besuch des Zustands $s$ noch erwartet werden, wenn die Aktionsauswahl mit der Policy $\pi$ geschieht.
Analog wird für den Zustand $s \in \mathcal{S}$ und eine Aktion $a \in \mathcal{S}$ unter einer Policy $\pi$ der Aktionswert als
\begin{equation}
	Q_\pi(s, a) := \mathbb{E}_\pi\left[G_t | S_t = s, A_t = a\right]
\end{equation}
definiert, also die erwartete (diskontierte) Summe an Belohnungen nach dem Besuch von $s$ und der Wahl von Aktion $a$, wenn danach der Policy $\pi$ gefolgt wird.
Zustands- und Aktionswerte sind zentrale Begriffe im Reinforcement Learning, die in vielen Algorithmen verwendet werden.
Das Ziel, den Return $G_0$ zu maximieren, lässt sich auch äquivalent als Maximierung von $V(S_0)$ formulieren.

\section{Exakte Lösungen mittels Dynamic Programming}
\label{section:dp}
Ein klassisches Verfahren für das Lösen von Markov-Entscheidungsprozessen ist die dynamische Programmierung.
Das setzt voraus, dass sowohl Zustands- als auch Aktionsraum endlich und die Übergangs- und Belohnungswahrscheinlichkeiten $p$ bekannt sind.
Anschließend wird eine angenäherte Funktion für die Zustandswerte $V^*$ als Array gespeichert und mit beliebigen Werten initialisiert.
Dann werden rundenweise für alle Zustände $s\in\mathcal{S}$ ihre Einträge in dieser Tabelle mithilfe der Bellman-Gleichung
\begin{equation}
	V^*(s) \leftarrow \max_{a\in\mathcal{A}}{\left(\sum_{s'\in\mathcal{S}, r \in \mathbb{R}}{p_{s, a}\left(s', r\right) \left(r + V^*(s')\right)}\right)}
\end{equation}
als Berechnungsvorschrift aktualisiert.
Unabhängig davon, ob diese Berechnung in-place oder mit zwei alternierenden Arrays ausgeführt wird, konvergiert sie zur Zustandswertefunktion einer optimalen Policy.
Bei \cite{Sutton1998} ist das Verfahren als Wertiteration zu finden und ist in analoger Weise auch mit Aktionswerten möglich.
Die optimale Policy ist dann die deterministische, sogenannte Greedy-Policy.
Sie wählt in jedem Zustand das Argmax der Aktionen.

Dynamische Programmierung hat allerdings starke Forderungen, die in der Praxis schwer erfüllbar sind.
Erstens müssen die Wahrscheinlichkeitsverteilungen der Belohnungen und Zustandsübergänge bekannt sein.
Die sind aber unter anderem bei dem in dieser Arbeit untersuchten Markt nicht explizit vorhanden.
Eine umfangreiche mathematische Analyse wäre erforderlich.

Dieses Problem lässt sich allerdings noch im Rahmen abgewandelter DP-Verfahren lösen.
Grundidee dieser sogenannten Monte-Carlo-Verfahren ist es, eine Tabelle für die Schätzung der Aktionswerte zu speichern und damit eine Greedy-Policy zu erstellen.
Es werden nach und nach Episoden generiert und mit den empirischen Daten die Schätzung der Aktionswerte verbessert.
In jedem Schritt wird dabei mit einer Wahrscheinlichkeit von $1-\varepsilon$ dieser Greedy-Policy gefolgt, und für die Exploration mit einer Wahrscheinlichkeit von $\varepsilon$ eine zufällige Aktion ausgewählt.
Weiteres dazu kann in \cite{Sutton1998} nachgelesen werden.

\section{Approximation als Ausweg bei zu großem Zustands- und Aktionsraum}
Doch auch dieses Verfahren kann auf den beschriebenen Markt nicht angewandt werden.
In jedem Fall müsste die Formulierung mit dem diskreten Aktionsraum gewählt werden.
Hindernis ist aber auch hier der große Zustands- und Aktionsraum: soll das Monte-Carlo-Verfahren angewandt werden, so ist dafür eine Tabelle der Größe $|\mathcal{S} \times \mathcal{A}|$ erforderlich.
Bei den in Tabelle \ref{tab:default_parameters} genannten Standardwerten müsste die Tabelle $50 \cdot 500 \cdot 100 \cdot 10^3 \cdot 100 \cdot 10^3 = 2.5 \cdot 10^{14}$ Einträge haben.
Obwohl dieser Wert bereits jeden verfügbaren Speicher sprengt, ist die Anzahl der unterschiedlichen Preisstufen mit 10 für realistische Verhältnisse recht niedrig angesetzt.
Tatsächlich liegt der benötigte Speicher in $\mathcal{O}\left(p_{max}^6\right)$, was eine Skalierung unmöglich macht.
Weil dies sogar nur die Betrachtung des Speichers war -- bis Konvergenz erreicht ist, müsste jeder Zustand sehr oft durchlaufen werden -- muss geschlussfolgert werden, dass eine Lösung, die für alle Zustände die Zustandswerte ausrechnen möchte, nicht realisierbar ist.

Approximationen sind also unumgänglich.
Sie beruhen auf der Einsicht, dass der Zustandsraum zwar sehr groß ist, aber viele Zustände sich sehr ähnlich sind.
Dadurch können Modelle mit deutlich weniger Parametern als Zuständen diese gewünschten Funktionen  mit für praktische Zwecke ausreichender Genauigkeit annähern.
Beispiele und Erklärungen dazu sind auch in \cite{lapan2020deep} zu finden.
Alle folgenden Verfahren nutzen sogenannte künstliche neuronale Netze, die am weitesten verbreitete Technologie, um komplexe nichtlineare Funktionen zu approximieren.

\section{Q-Learning-Verfahren}
\label{section:qlearning}
Ein weit verbreitetes Verfahren zum Lösen von Markov-Entscheidungsprozessen ist das sogenannte Q-Learning.
Die Idee ist hierbei, wie in dem in Abschnitt \ref{section:dp} beschriebenen Monte-Carlo-Verfahren, Aktionswerte einer optimalen Policy zu schätzen.
Dazu wird als Berechnungsvorschrift die Bellman-Gleichung verwendet.
Allerdings werden nicht unbedingt vollständige Episoden betrachtet, sondern der Aktionswert nach dem Folgezustand ergibt sich mittels der momentanen Schätzfunktion.
Diese Technik, zur Verbesserung von Schätzwerten rekursiv bisherige Schätzwerte zu verwenden, wird Bootstrapping genannt und wird laut \cite{Sutton1998} als wichtige Technologie angesehen.
Nach dem Training wird wieder die Greedy-Policy verwendet.
Während des Trainings wird eine $\varepsilon$-Soft-Policy genutzt, um für ausreichende Exploration zu sorgen.

Dieses Verfahren lässt sich mit Tabellen oder neuronalen Netzwerken umsetzen.
Letzteres wird \textit{Deep-Q-Learning} genannt.
\textit{Deep-Q-Networks} (DQNs) sind für diskrete Aktionsräume entworfen worden.
Die Schätzung der Aktionswertefunktion wird dabei so umgesetzt, dass an den Input des Netzes nur der Zustand angelegt wird und dann in einer Berechnung auf so viele Ausgabeneuronen abgebildet wird, wie es Aktionen gibt.
So erhält man bei nur einem der aufwendigen Läufe durch das neuronale Netz bereits alle Aktionswerte auf einmal.
Die Wahl des Maximums und des Argmax sind dann unmittelbar möglich.
Das Training findet iterativ statt, indem die bisherigen Schätzwerte des Netzes mit Zustandsübergängen $(s, a, r, s')$ aus der gesammelten Erfahrung $\mathcal{D}$ in Richtung des Zielwertes $r+\gamma \max_{a'\in\mathcal{A}}{Q(s',a')}$ bewegt werden.
In den heutigen Deep-Learning-Frameworks wie PyTorch \cite{NEURIPS2019_9015} wird üblicherweise mit einer Verlustfunktion gearbeitet, die dann automatisch differenziert wird.
Mit einem Optimizer wie Stochastic Gradient Descent oder Adam \cite{adam2014} wird dieser Verlust dann mithilfe des Gradienten minimiert.
Diese Verlustfunktion beim Deep-Q-Learning lautet für eine Schätzfunktion der Aktionswerte mit den Parametern $\psi$
\begin{equation}
	L_\mathcal{D}(\psi) = \mathbb{E}_{(s, a, r, s')\sim\mathcal{D}}\left[\left(Q_\psi(s,a)-\left(r+\gamma \max_{a'\in\mathcal{A}}{Q_\psi(s',a')}\right)\right)^2\right].
\end{equation}
Erfolgreiche Q-Learning-Umsetzungen verwenden einen Experiencebuffer, in dem die Erfahrung $\mathcal{D}$ gespeichert wird.
Weiterhin speichern sie zwei Netze: neben dem, das trainiert wird, auch ein sogenanntes Zielnetz, was in der Schätzung des Zielwertes verwendet wird.
Das verhindert ungewollte Oszillationen und Instabilität, wie in \cite{atari2014} beschrieben.
Zahlreiche weitere Techniken zur Verbesserung von Deep-Q-Networks sind verbreitet und verbessern das Training erheblich.
Zahlreiche von ihnen sind in \cite{qlearningcomparison2017} aufgeführt.
Für die gleich folgenden Verfahren ist insbesondere die Erkenntnis wichtig, dass Q-Learning-Verfahren zur systematischen Überschätzung der Aktionswerte neigen.
Näheres dazu steht in \cite{qlearningoverestimation}.

Q-Learning-Verfahren sind Off-Policy-Algorithmen.
Das bedeutet, dass ihr Training auch mit Zustandsübergängen stattfinden kann, die nicht von der aktuellen Policy erzeugt wurden.
Der Grund dafür ist, dass bei einer perfekten Policy für \textbf{jeden} Zustandsübergang $(s, a, r, s')$ die Bellman-Gleichung
\begin{equation}
	Q(s, a) = \sum_{s'\in\mathcal{S}, r \in \mathbb{R}}{p_{s, a}\left(s', r\right) \left(r + \max_{a' \in \mathcal{A}}{Q(s', a')}\right)}
\end{equation}
erfüllt ist.
Das ist zwar bei Approximationslösungen nicht erreichbar, die Minimierung des Erwartungswert über die quadrierten Differenzen zeigt aber die Entfernung von diesem Ziel.

\section{DDPG und TD3}
Die Verallgemeinerung auf stetige Aktionsräume ist bei Deep-Q-Networks nicht grundlegende Designänderungen möglich.
Allerdings gibt es ein RL-Verfahren, Deep Deterministic Policy Gradient (DDPG), das in \cite{10.5555/3044805.3044850} publiziert wurde und als Verallgemeinerung von Q-Learning auf Umgebungen mit stetigen Aktionsräumen angesehen werden kann.
Es handelt sich dabei ebenfalls um ein Off-Policy-Verfahren.
Ein neuronales Netz mit den Parametern $\psi$ berechnet die Aktionswerteschätzfunktion $Q_\psi: \mathcal{S} \times \mathcal{A}\rightarrow \mathbb{R}$ (jetzt muss die Aktion mit in die Eingabe).
Um an die Policy zu gelangen, ist es dabei aber kaum möglich, das tatsächliche Maximum dieser Funktion $Q_\psi$ zu bestimmen.
Für neuronale Netze ist keine analytische Lösung möglich und numerische Verfahren sind viel zu aufwendig für die vielen benötigten Trainingsschritte.
Die Lösung besteht in einem weiteren neuronalen Netz mit den Parametern $\phi$, das explizit die deterministische Policyfunktion $\pi_\phi: \mathcal{S} \rightarrow \mathcal{A}$ berechnet, ein Actor-Netz.
Das Training des Q-Netzes findet wie beim DQN statt, außer dass für den Zielwert die Policyfunktion verwendet wird.
Die dazugehörige Verlustfunktion lautet sehr ähnlich zu der von Q-Learning
\begin{equation}
	L(\psi) = \mathbb{E}_{(s, a, r, s')\sim\mathcal{D}}\left[\left(Q_\psi(s,a)-\left(r+\gamma Q_{\psi'}(s',\pi_\phi(s'))\right)\right)^2\right].
\end{equation}
Dabei wird bei DDPG wird ein Zielnetz mit Parametern $\psi'$ für die Schätzung der Policyfunktion verwendet, was nach jedem Optimierungsschritt mit dem Hyperparameter $\rho \in [0, 1]$ und der Polyak-Mittelung $\psi' \leftarrow (1 - \rho) \psi' + \rho \psi$ aktualisiert wird.
Zur Optimierung der Policyfunktion in dem Sinne, dass sie für einen Zustand möglichst das Argmax berechnet, wird Gradientenanstieg verwendet.
Das wird dadurch möglich, dass beide neuronalen Netze differenzierbar sind und der Aktionsraum stetig ist.
So kann der Gradient $\nabla_\phi \mathbb{E}_{s\sim\mathcal{D}}\left[Q_\psi(s, \pi_\phi(s))\right]$ durch die automatische Differenzierung des Deep-Learning-Frameworks berechnet und ein Optimierungsschritt in die Richtung des größten Anstiegs vorgenommen werden.

Nachfolgende Analysen haben ergeben, dass die systematische Überschätzung der Funktionswerte bei DDPG sich besonders stark auf die Leistung auswirkt.
In dem Paper \cite{DBLP:journals/corr/abs-1802-09477} wurden deshalb drei Verbesserungen für den DDPG-Algorithmus vorgeschlagen:
\begin{enumerate}
	\item Es werden zwei separate Netze für die Aktionswertefunktion gehalten.
	Für die Berechnung des Zielwertes wird immer das Minimum von beiden genommen.
	\item Weil die Q-Netze manchmal kleinere Ausbuchtungen haben, die vom Gradientenanstieg ausgenutzt werden, wird bei der Berechnung des Zielwertes der Aktionsauswahl durch das Policynetz normalverteiltes Rauschen angefügt.
	Die Aktionen werden dabei selbstverständlich auf den Aktionsraum geclippt.
	\item Im Gegensatz zu vielen DDPG-Implementierungen wird das Policy-Netz nur alle zwei Q-Netz-Updates trainiert.
\end{enumerate}
Der entstehende Algorithmus wird als TD3 bezeichnet und hat bei vielen Tests durch bessere Performance überzeugt.

\section{Policy-Gradient-Verfahren}
Eine alternative Herangehensweise zu den wertebasierten Verfahren sind die Policy-Gradient-Verfahren.
Während wertebasierte Verfahren zunächst Aktions- oder Zustandswerte schätzen, und daraus ihre Policy herleiten, berechnen Policy-Gradient-Verfahren direkt eine stochastische Policyfunktion $\pi_\phi$ z.B. mit einem neuronalen Netz.
Sie verfolgen das gleiche in Abschnitt \ref{section:mdp_fundamentals} formulierte Ziel, nämlich eine Policy zu finden, die den Zustandswert des Startzustandes und damit den erwarteten Return einer Episode zu maximieren.
Das ursprüngliche Verfahren dieser Klasse heißt REINFORCE \cite{10.1007/BF00992696} und ist aus dem Jahr 1992.
Danach wurden zahlreiche Verbesserungen vorgenommen, die zu A2C und PPO führen.
Grundidee ist hierbei, Gradientenanstieg für die Anpassung der Policy zu nutzen, sodass der Zustandswert des Startzustandes maximiert wird.
Dazu gibt es ein bedeutendes theoretisches Resultat, das eine Möglichkeit aufzeigt, den Gradienten aus gesampelten Episoden leicht zu schätzen, das sogenannte Policy-Gradient-Theorem.
Es lässt sich für den REINFORCE-Algorithmus als
\begin{equation}
	\nabla_\phi V_{\pi_\phi}(S_0) \propto \mathbb{E}_{S_t, A_t}\left[G_t \nabla\log{\pi_\phi(S_t, A_t)}\right]
\end{equation}
formulieren, wobei $S_t$ und $A_t$ als Zufallsvariablen den Zustand und die Aktion im $t$-ten Schritt angeben, und $G_t$ den Result vom $t$-ten Schritt aus.
Der REINFORCE-Algorithmus sampelt nun Episoden unter Verwendung einer Policy und nutzt eine Schätzung dieses Gradienten für Gradientenanstieg.
Die Updateregel ist intuitiv: Aktionen, auf denen ein positivem Return folgt, werden wahrscheinlicher gemacht und Aktionen mit negativem Return unwahrscheinlicher.

Doch REINFORCE hat für die praktische Anwendung einen offensichtlichen Nachteil: Die Bewertung, ob eine Aktion >>gut<< oder >>schlecht<< ist, richtet sich immer unmittelbar nach dem gemessenen Return.
In einer Umgebung, die nur negative Rewards vergibt, wird der Gradient immer negativ sein und damit auch für Aktionen, die in der Situation deutlich besser sind als andere.
Abhilfe schafft hier die sogenannte Baseline, eine Funktion $b: \mathcal{S} \rightarrow \mathbb{R}$, die einen Vergleichswert für Zustände angibt.
Sie ist sinnvollerweise eine Schätzung des Zustandswertes.
Mit ihr wird bewertet, ob ein erreichter Return für einen Zustand gut oder schlecht ist.
Anstelle des Returns wird dann für den Gradienten der Advantagewert $\hat{A}_t := G_t - b(S_t)$ verwendet.
Algorithmen, die nach diesem Prinzip funktionieren, werden Actor-Critic-Algorithmen genannt.
Das ist eine Metapher für die zwei Netze Actor (berechnet die Policy) und Critic (schätzt den Zustandswert und ermöglicht damit die >>Kritik<< einer Aktion).
In manchen Implementierungen teilen sich die beiden Netze vordere Schichten.

Um aus den Trainingsdaten den Advantagewert zu schätzen, gibt es mehrere Herangehensweisen.
Eine Möglichkeit ist es, die Episoden bis zum Ende zu sampeln und die dabei gemessenen Werte zu verwenden.
Das ist ein Schätzer ohne Bias, aber mit großer Varianz.
Alternativ kann Bootstrapping verwendet werden, was die Varianz reduziert aber Bias einführt.
Um einen Kompromiss zwischen Bias und Varianz zu finden, wurde der Generalized Advantage Estimator \cite{https://doi.org/10.48550/arxiv.1506.02438} vorgeschlagen, der auch in diesen Implementierungen verwendet wird.

Im Gegensatz zu den Q-Learning-Verfahren aus dem Abschnitt \ref{section:qlearning} handelt es sich bei dieser Gruppe um On-Policy-Algorithmen.
Das bedeutet, dass die Trainingsdaten mit der Policy gewonnen werden mussten, die auch trainiert wird.
Training mit älteren Daten ist nicht möglich.

Ein Problem bei Actor-Critic-Methoden ist, dass abrupte Änderungen der Policy die Besuchswahrscheinlichkeiten der Zustände radikal verändern kann, ein Grund für Instabilität.
Um das Problem zu lindern, soll beim Lernen die Veränderung der Wahrscheinlichkeiten der Policy gering gehalten werden.
Ein wichtiger Beitrag dazu war TRPO, das die Kullback-Leibler-Divergenz zwischen alten und neuen Wahrscheinlichkeiten begrenzt. \cite{trpo2015}
Sein Nachfolger PPO \cite{ppo2018} begrenzt den Unterschied zwischen alten und neuen Wahrscheinlichkeiten auf einfachere Weise. 
Dazu wird der Gradientenanstieg nach dem Sammeln von Erfahrung in kleinere Schritte zerlegt und die alten Parameter $\phi_{old}$ gespeichert.
Das geschieht, um für eine Aktion $a \in \mathcal{A}$ in einem Zustand $s \in \mathcal{S}$ das Verhältnis aus neuer und alter Wahrscheinlichkeit $r(t, \phi, \phi_{old}) := \pi_\phi(S_t, A_t) / \pi_{\phi_{old}}(S_t, A_t)$ berechnen zu können.
Damit wird mit dem Hyperparameter $\varepsilon \in [0, 1]$ die Zielfunktion
\begin{equation}
	L_{\phi_{old}}(\phi) = \mathbb{E}_{S_t, A_t}\left[\min{\left(\hat{A}_t \text{clip}{\left(1 - \varepsilon, r(t, \phi, \phi_{old}), 1 + \varepsilon\right)}, \hat{A}_t r(t, \phi, \phi_{old})\right)}\right]
\end{equation}
definiert, um sie zu maximieren.
Idee dahinter ist, die Wahrscheinlichkeit guter Aktionen -- egal wie gut sie waren -- um höchstens den Faktor $\varepsilon$ zu verbessern.
Die Absenkung der Wahrscheinlichkeit für schlechte Aktionen soll auf die gleiche Art begrenzt werden.
Hat sich die Wahrscheinlichkeit einer guten Aktion bereits um mehr als den Faktor $\varepsilon$ erhöht (bzw. der einer schlechten Aktion verringert), so fällt der Gradient auf null.
Dass nicht nur geclippt, sondern auch minimiert wird, behandelt den Fall, dass die Wahrscheinlichkeit einer guten Aktion gesenkt bzw. die einer schlechten Aktion erhöht wurde.
Gerät diese Wahrscheinlichkeit aus dem Vertrauensbereich (in die falsche Richtung) heraus, so soll dies weiterhin korrigiert werden.

\section{Soft Actor Critic}
\label{section:sac}
Eines der neuesten Verfahren ist Soft Actor Critic (SAC) \cite{haarnoja2018soft}. 
Es kann als Verallgemeinerung von TD3 mit stochastischer Policy angesehen werden.
Eine zentrale Änderung gegenüber den anderen Verfahren ist die Umformulierung des RL-Ziels, den erwarteten Return einer Episode zu maximieren.
Es wird darum ergänzt, dass bei jedem Schritt die Entropie $\mathcal{H}$ des Actors hoch sein soll.
Das bedeutet, dass eine Policy $\pi$ gefunden werden soll, die möglichst zufällig arbeitet und dennoch sehr gute Ergebnisse liefert.
Mit dem Parameter $\alpha$ wird die Balance aus Entropie und Belohnungen eingestellt.
Die Formulierung dieses >>soften<< Returns lautet:
\begin{equation}
	G_{soft,t} := \sum_{i=t}^{n_{ep} - 1} \gamma^{i - t} \left(R_i + \alpha \mathcal{H}(\pi(\cdot | S_i))\right)
\end{equation}
Das Training der Aktionswertefunktion findet genau wie bei TD3 statt.
Die Verlustfunktion zur Optimierung der Policy schreibt sich mit folgendem Ausdruck:
\begin{equation}
	L_{Q_\theta}(\phi) = \mathbb{E}_{S_t \sim \mathcal{D}} \left[D_{KL}\left(\pi_\phi(\cdot \mid S_t) | \frac{\exp(Q_\theta(S_t, \cdot))}{Z_\theta(S_t)}\right)\right]
\end{equation}
Es wird dabei aus den Aktionswerten eine Wahrscheinlichkeitsverteilung gebaut, indem sie exponenziert und durch einen Normierungsfaktor $Z_\theta(S_t)$ geteilt werden.
Dieser Normierungsfaktor wird nur mathematisch dafür benötigt, dass das Integral über den gesamten Aktionsraum 1 ergibt.
Für den Algorithmus wird er aber nicht berechnet, da er nicht von $\theta$ abhängt und deshalb bei der Gradientenberechnung in die Lernrate hineingezogen wird.
Diese konstruierte Wahrscheinlichkeitsverteilung gewichtet wie erwartet die Aktionen mit hohem Aktionswert höher.
In diese Richtung soll dann die Policy verändert werden.
Im Programm muss aber diese komplizierte Verlustfunktion nicht abgeleitet werden, sondern lässt sich mit einigen Umformungen deutlich leichter formulieren.
Der Entropieparameter $\alpha$ kann entweder manuell eingestellt oder wie in \cite{https://doi.org/10.48550/arxiv.1812.05905} erklärt während des Trainings automatisch erlernt werden.

Soft Actor Critic ist angetreten, um bei der höheren Sample-Effizienz eines Off-Policy-Algorithmus weiterhin die Stabilität der On-Policy-Algorithmen zu haben.
Mit Soft Actor Critic konnten schwierige Aufgaben in verschiedenen Bereichen gelöst werden.
Nach dem aktuellen Stand der Forschung ist Soft Actor Critic einer der beliebtesten RL-Algorithmen für den kontinuierlichen Aktionsraum.
