\section{Hyperparameter}
\begin{table}[t]
    \centering
    \begin{tabular}{l l l}
        \toprule
        Symbol        & Erklärung                                                                   & Standardwert\\\midrule
        $n_{ep}$      & Anzahl der Schritte in einer Episode                                        & $500$\\
        $p_{max}$     & maximal wählbarer Preis für beide Agenten                                   & $10$\\
        $p_{einkauf}$ & Einkaufs- oder Produktionspreis für Neuprodukte                             & $3$\\
        $p_{lager}$   & Preis pro eingelagertem Gebrauchtprodukt pro Schritt                        & $0.1$\\
        $k$           & Kundenzahl, die pro Schritt den Markt besucht                               & $20$\\
        $m_{lager}$   & maximale Anzahl von Gebrauchtprodukten, die im Lager gehalten werden können & $100$\\
        $c$           & Anteil der Eigentümer, die pro Schritt einen Rückverkauf erwägen            & $0.05$\\\bottomrule
    \end{tabular}
    \caption{Marktparameters mit kurzer Erklärung und für diese Experimente verwendete Standardwerte}
    \label{tab:default_parameters}
\end{table}

\begin{table}[t]
    \centering
    \begin{tabular}{p{0.5\textwidth} l}
        \toprule
        Parameter                                     & Wert\\\midrule
        Lernrate                                      & $10^{-3}$\\
        Größe des Experiencebuffers                   & $10^6$\\
        Episoden bis zum Beginn des Lernens           & $100$\\
        Größe eines Minibatches                       & $100$\\
        Koeffizient für die Polyak-Mittelung ($\tau$) & $0.005$\\
        Diskontierungsfaktor ($\gamma$)               & $0.99$\\\bottomrule
    \end{tabular}
    \caption{Hyperparameter für DDPG und TD3}
    \label{tab:DDPGHyperparameters}
\end{table}

\begin{table}[t]
    \centering
    \begin{tabular}{p{0.5\textwidth} l}
        \toprule
        Parameter                                     & Wert\\\midrule
        Lernrate                                      & $7\cdot 10^{-4}$\\
        Anzahl der Schritte pro Update                & $5$\\
        Diskontierungsfaktor ($\gamma$)               & $0.99$\\\bottomrule
    \end{tabular}
    \caption{Hyperparameter für Advantage Actor Critic}
    \label{tab:A2CHyperparameter}
\end{table}

\begin{table}[t]
    \centering
    \begin{tabular}{p{0.5\textwidth} l}
        \toprule
        Parameter                                     & Wert\\\midrule
        Lernrate                                      & $3 \cdot 10^{-4}$\\
        Anzahl der Schritte pro Update                & $2048$\\
        Größe eines Minibatches                       & $64$\\\
        Anzahl der Epochen pro Update                 & $10$\\
        \textit{clip\_range} ($\varepsilon$)          & $0.2$\\
        Diskontierungsfaktor ($\gamma$)               & $0.99$\\\bottomrule
    \end{tabular}
    \caption{Hyperparameter Proximal Policy Optimization}
    \label{tab:PPOHyperparameters}
\end{table}

\begin{table}[t]
    \centering
    \begin{tabular}{p{0.5\textwidth} l}
        \toprule
        Parameter                                     & Wert\\\midrule
        Lernrate                                      & $3 \cdot 10^{-4}$\\
        Größe des Experiencebuffers                   & $10^6$\\
        Episoden bis zum Beginn des Lernens           & $100$\\
        Größe eines Minibatches                       & $256$\\\
        Entropiekoeffizient ($\alpha$)                & automatisch\\
        Koeffizient für die Polyak-Mittelung ($\tau$) & $0.005$\\
        Diskontierungsfaktor ($\gamma$)               & $0.99$\\\bottomrule
    \end{tabular}
    \caption{Hyperparameter Soft Actor Critic}
    \label{tab:SACHyperparameters}
\end{table}

\section{Definition der regelbasierten Strategien}
\label{section:rulebased_definition}
Der Neupreis wird als
\begin{equation}
	\pi\left(p_{2, neu}\right)_{neu} = \max{\left(p_{2, neu} - 1, p_{einkauf} + 1\right)}
\end{equation}
gesetzt.
Beim Gebraucht- und Rückkaufpreis wird je nach Lagerstand entschieden.
Der Gebrauchtpreis wird in Abhängigkeit des Konkurrenzpreises und des Lagerstandes folgendermaßen gesetzt:
\begin{equation}
	\pi\left(p_{2, gebraucht}, n_{lager}\right)_{gebraucht} =
	\begin{cases}
		p_{2, gebraucht} + 1 & n_{lager} < m_{lager} / 15\\
		p_{2, gebraucht} - 1 & n_{lager} < m_{lager} / 8\\
		p_{2, gebraucht} - 2 & \text{sonst}
	\end{cases}
\end{equation}
Beim Rückkaufpreis werden die gleichen Fallunterscheidungen unternommen und der Preis gesetzt als:
\begin{equation}
	\pi(p_{2, re}, n_{lager})_{re} =
	\begin{cases}
		p_{2, re} + 1 & n_{lager} < m_{lager} / 15\\
		p_{2, re} - 1 & n_{lager} < m_{lager} / 8\\
		p_{2, re} - 2 & \text{Sonst}
	\end{cases}
\end{equation}
Alle Preise werden auf den Aktionsraum beschränkt (für den Fall, dass die Rechnung Ergebnisse kleiner als null oder größer als $p_{max}$ ermittelt).
