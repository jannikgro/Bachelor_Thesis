Während Pricing ein altes Problem ist, war die Forschungsaktivität zu Pricing mit intensivem Wettbewerb und gerade auf Online-Marktplätzen bis vor einigen Jahren auf einem geringen Level, wie eine Übersichtsarbeit aus dem Jahre 2022 zeigt. \cite{Gerpott2022}
Demnach dominierten Monpolszenarien oder Szenarien mit starken, aber einschränkenden Annahmen.

Im Bereich der algorithmen- und datengetriebenen Preisoptimierung lässt sich die Schwierigkeit in zwei separate Herausforderungen unterteilen: Das Schätzen des Kundenverhaltens und das Optimieren des Pricings bei bekanntem oder erlerntem Kundenverhalten.
Schlosser und Boissier \cite{10.1145/3219819.3219833} lösen diese Probleme einzeln.
Zunächst werden Regressionsverfahren angewendet, um das Kundenverhalten zu erlernen und anschließend ein approximiertes Dynamic Programming Verfahren vorgestellt, um trotz des >>Fluchs der Dimension<< Preisstrategien optimieren zu können.
Die mit diesem Verfahren optimierten Preisstrategien konnten erfolgreich auf Amazon Marketplace angewandt werden.

Mit einem sehr ähnlichen Marktmodell haben Kastius und Schlosser \cite{Kastius2022} Reinforcement Learning eingesetzt, um unter Wettbewerbsbedingungen die Preisstrategie zu optimieren.
Untersucht wurde eine \textit{Linear Economy} im Duopol und Oligopol.
Als RL-Verfahren wurden Deep-Q-Networks (mit diskretem Aktionsraum) und Soft Actor Critic ausprobiert.
Dabei schnitt Soft Actor Critic erheblich besser ab, aber die DQNs konnten die Aufgabe auch lösen.
Es wurde weiterhin festgestellt, dass im Duopol gelegentlich Kartelle gebildet werden, wenn die Kaufbereitschaft der Kunden bei höheren Preisen nicht begrenzt wird.