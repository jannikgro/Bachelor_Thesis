\section{Recommerce -- ein Überblick}
In dieser Arbeit werde ich ein Lehrbuch \cite{Sutton1998} und ein Paper \cite{DBLP:journals/corr/abs-1712-01815} zitieren.

\section{Der Markt als Markov-Entscheidungsprozess}
Der im vorigen Kapitel intuitiv umrissene Markt wird hier als Markov-Entscheidungsprozess modelliert.
Dazu wird das Marktgeschehen eines geschäftlich relevanten Zeitraumes, z.B. eines Tages oder eines Monats in $n_{ep}$ gleich lange Zeitabschnitte unterteilt.
In der Terminologie des Reinforcement Learnings wird ein einzelner Abschnitt als Schritt, die Gesamtheit als Episode bezeichnet.
Vor jedem Schritt wird dem Agenten der Zustand des Marktes präsentiert.
Der Reinforcement-Learning-Agent setzt in Abhängigkeit des Zustandes drei Preise:
\begin{enumerate}
    \item für gebrauchte Produkte der Produktlinie ($p_{1, gebraucht}$),
    \item für diese Produkte in neu ($p_{1, neu}$) sowie
    \item den Rückkaufpreis ($p_{1, re}$), für den von einem Eigentümer gebrauchte Produkte zurückgekauft werden.
\end{enumerate}
Alle drei Preise bewegen sich zwischen 0 und dem maximal möglichen Preis $p_{max}$.
Damit ist der (stetige) Aktionsraum als $\mathcal{A}=[0, p_{max}]^3$ definiert.
Der Konkurrent legt ebenfalls diese drei Preise für seine Produktlinie fest ($p_{2, gebraucht}$, $p_{2, neu}$, $p_{2, re}$).

In jedem Schritt besuchen $k \in 2\mathbb{N}$ Kunden den Marktplatz, betrachten die Angebote und treffen eine von fünf möglichen Entscheidungen.
Entweder kaufen sie kein Produkt, das gebrauchte oder neue beim ersten Anbieter (RL-Agent) oder das gebrauchte oder neue beim Konkurrenten.
Das Kaufverhalten ist aber keinesfalls deterministisch.
Es wird stark durch die Preise beeinflusst, aber unterschiedliche Kunden haben unterschiedliche Erwartungen, unterschiedliche Anbieterpräferenzen oder unterschiedliche Einstellungen zu Nachhaltigkeit.
Deshalb wird eine Wahrscheinlichkeitsverteilung des Kaufverhaltens über die Kundschaft erstellt, aus der dann für einzelne Kunden geshuffelt wird.
Das geschieht, indem für die einzelnen Optionen Präferenzen aufgestellt werden, aus denen dann per Softmax die Zähldichte der Wahrscheinlichkeitsverteilung berechnet wird.
Die Berechnungsformel für den Präferenzvektor lautet:
\begin{equation}
    \sigma_{kunde}(p_{1, gebraucht}, p_{1, neu}, p_{2, gebraucht}, p_{2, neu}) =
    \begin{pmatrix}
        1\\
        \frac{5.5}{p_{1, gebraucht}} - \exp{(p_{1, gebraucht} - 0.5 p_{max})}\\
        \frac{10}{p_{1, neu}} - \exp{(p_{1, neu} - 0.8 p_{max})}\\
        \frac{5.5}{p_{2, gebraucht}} - \exp{(p_{2, gebraucht} - 0.5 p_{max})}\\
        \frac{10}{p_{2, neu}} - \exp{(p_{2, neu} - 0.8 p_{max})}\\
    \end{pmatrix}
\end{equation}
Die Präferenz für eines der Angebote basiert also hauptsächlich auf dem Preis-Leistungs-Verhältnis.
Gebrauchten Produkten wird dabei 55\% der Wertschätzung gegenüber Neuprodukten entgegengebracht.
Zusätzlich sinkt die Präferenz deutlich, wenn der Preis $0.5 p_{max}$ bei Gebraucht- und $0.8 p_{max}$ bei Neuware überschreitet.
Das modelliert die grundsätzliche Zahlungsbereitschaft der Kunden für diese Produkte und verhindert einen Preiszyklus nach oben.
Weiterhin werden die Produkte der beiden Anbieter als gleichwertig betrachtet.
So ist der Wettbewerb zwischen den Anbietern symmetrisch und die Gewinne der beiden Anbieter vergleichbar.
Der Wert 1 als Präferenz für das Nichtkaufen ist festgelegt.
Fallen die Präferenzen der anderen Optionen niedrig aus, so wird durch die Softmax-Funktion eine hohe Wahrscheinlichkeit auf das Nichtkaufen entfallen.
Darstellung [Test] veranschaulicht das Kundenverhalten an einigen Beispielen.

Für jeden der beiden Anbieter enthält das Marktmodell einen Zähler für die Anzahl der im Lager befindlichen Gebrauchtprodukte, $n_{lager}$.
Das Lager hat ein Fassungsvermögen von $m_{lager}$ Produkten.
Der initiale Lagerstand wird für beide Anbieter unabhängig und uniform zufällig zwischen $0$ und $m_{lager}$ gewählt.
Wenn mehr Produkte zurücklaufen, werden diese verworfen und der Lagerstand bleibt beim Maximum.
Kauft ein Kunde ein gebrauchtes Produkt, obwohl das Lager leer ist, so muss der Anbieter Entschädigung zahlen.
Die Entschädigung ist auf $2 p_{max}$ pro enttäuschtem Kunden festgelegt.
Das erzeugt einen starken Anreiz für die Anbieter, das Lager nicht leerlaufen zu lassen.
Allerdings verursacht auch die Lagerhaltung Kosten: Am Ende eines jeden Schrittes werden Kosten von $p_{lager}$ pro eingelagertem Element berechnet.
Das wiederum treibt die Anbieter dazu, ihr Lager möglichst klein zu halten und stellt sie vor die Herausforderung, mit dem Rückkaufpreis nicht nur auf ihren Konkurrenten zu reagieren, sondern auch ihr Lager zu regulieren.

Weiterhin verwaltet der Markt einen Zähler für die Anzahl der Produkte, die sich in Zirkulation befinden.
Er wird als $n_{zirk}$ bezeichnet und uniform zufällig zwischen $0$ und $5 m_{lager}$ initialisiert.
Wird ein Neuprodukt gekauft, so wird dieser Zähler um eins erhöht.
Der Kauf von bereits gebrauchten Produkten führt nicht zu einer Erhöhung.
Diese Circular Economy modelliert also nur den einmaligen Weiterverkauf von Produkten.

Je mehr Kunden im Besitz eines Produkts sind, desto mehr dieser Eigentümer erwägen, ihr Produkt nicht länger besitzen zu wollen.
Sie stehen in der Zirkularwirtschaft vor einer Entscheidung aus vier Möglichkeiten.
Sie können ihr Produkt weiter behalten, es wegwerfen oder an einen der beiden Anbieter zurückverkaufen.
Insgesamt stellen sich in jedem Schritt $c \cdot n_{zirk}$ Eigentümer dieser Entscheidung.
Nach dem gleichen Verfahren wie bei der Kaufentscheidung der Kunden wird ihr Verhalten mit einer diskreten Wahrscheinlichkeitsverteilung modelliert, aus der dann für jeden der $c \cdot n_{zirk}$ Eigentümer geshuffelt wird.
Die Berechnung der Präferenzen der Eigentümer geschieht auf Grundlage der Preise der Anbieter mittels dieser Formel:
\begin{equation}
    \sigma_{eigentuemer}=
    \begin{pmatrix}
        1\\
        \min \left(20, \frac{2}{p_{1, gebraucht}},  \frac{2}{p_{2, gebraucht}}\right)\\
        2 \exp{\left(\frac{p_{1, re} - \min{(p_{1, gebraucht}, p_{1, neu})}}{\min{(p_{1, gebraucht}, p_{1, neu})}}\right)}\\
        2 \exp{\left(\frac{p_{2, re} - \min{(p_{2, gebraucht}, p_{2, neu})}}{\min{(p_{2, gebraucht}, p_{2, neu})}}\right)}\\
    \end{pmatrix}
\end{equation}
Diese Formel setzt die Präferenz für das Halten eines Produktes auf $1$ fest.
Die Präferenz für das Wegwerfen wird höher, wenn der Gebrauchtpreis niedriger ist.
Die Motivation dazu ist, dass bei einem geringen Kaufpreis der Kunde das Produkt weniger wertschätzt und daher eher geneigt ist, es wegzuwerfen.
Die Präferenzen für das Zurückverkaufen an die jeweiligen Anbieter werden deutlich höher, wenn der prozentuale Unterschied zwischen dem Rückkaufpreis nah am Verkaufspreis ist.
Gedanke dahinter: Die Kunden möchten ihrem Anbieter keine große Rendite zugestehen.
Die Zähldichte der Wahrscheinlichkeitsverteilung ergibt sich dann wieder als Softmax dieser Präferenzen.

Auf einem Online-Marktplatz reagieren die Anbieter gegenseitig auf Preisveränderungen.
Während der Agent seine Preise stets zu Beginn jedes Schrittes setzt, reagiert der Wettbewerber in der Mitte eines Schrittes auf den Agenten.
Das bedeutet, dass nach $k/2$ Kunden die Wahrscheinlichkeitsverteilungen für Kunden- und Eigentümerverhalten neu berechnet wird.
So entsteht ein symmetrischer und fairer Markt, in dem für jeden Anbieter nach dem Setzen seiner Preise stets $k/2$ Kunden auf das Angebot treffen, das neben den neuen Preisen noch die vorigen des Konkurrenten erhält.
Weitere $k/2$ Kunden treffen dann auf diese Preise und die erneute Reakion des Wettbewerbers.
Einzige Ausnahme ist hier der erste und der letzte Schritt.
Weil der Prozess erst in Gang gesetzt werden muss, muss der Konkurrent den ersten halben Schritt mit Defaultwerten verbringen.
Im Gegenzug wird er bei seinen Preisen im letzten Schritt (die dann noch für einen halben Schritt gelten) keine Preisreaktion mehr bekommen.
Diese Randeffekte fallen jedoch bei den gewählten Episodenlängen nicht ins Gewicht.

Dieser Markt erhält also abwechselnd von zwei konkurrierenden Anbietern Aktionen und kann auch für beide Anbieter Gewinne ausrechnen.
Zum Markov-Entscheidungsprozess wird dieser Markt, wenn er aus der Sicht eines der beiden Anbieter betrachtet wird.
Der andere Anbieter wird damit als Teil der Umgebung betrachtet.
So wurde bereits oben vom Agenten und vom Konkurrenten geschrieben.
Das Verhalten des Konkurrenten kann deterministisch oder stochastisch sein, muss aber für die Markov-Eigenschaft unveränderlich sein.

Die Belohnung, die der Agent in einem Schritt erhält, ist die Differenz aus Einnahmen und Ausgaben des Agenten in diesem Schritt.
Mit dem Verkauf neuer oder gebrauchter Ware erzielt der Agent Einnahmen.
Ausgaben sind nicht nur der bereits erklärte Rückkaufpreis und die Lagerkosten und -strafen, sondern auch der Einkaufspreis für Neuware.
Pro Neuverkauf zahlt der Agent $p_{einkauf}$.
Damit muss für Gewinn auf dem Bereich der Neuware auf jeden Fall $p_{1, neu} > p_{einkauf}$ gelten.

Nun kann der Zustandsraum des Markov-Entscheidungsprozess $\mathcal{S}$ definiert werden.
Er ist siebendimensional und enthält Folgendes:
\begin{enumerate}
    \item die Nummer des aktuellen Schrittes in der Episode,
    \item die Anzahl der gerade zirkulierenden Elemente ($n_{zirk}$),
    \item die Anzahl der Elemente im eigenen Lager,
    \item den Preis des gebrauchten Produkts des Konkurrenten ($p_{2, gebraucht}$),
    \item den Neupreis des Konkurrenten ($p_{2, neu}$),
    \item den Rückkaufpreis des Konkurrenten ($p_{2, re}$) und
    \item den Lagerstand des Konkurrenten.
\end{enumerate}
Diese Werte beeinflussen das Verhalten der Kunden und des Konkurrenten.
Sie müssen gegeben sein, damit die sogenannte Markoveigenschaft erfüllt ist.
Diese besagt, dass die Wahrscheinlichkeiten der Zustandsübergänge und der Belohnungen nur vom aktuellen Zustand und der gewählten Aktion abhängen und insbesondere stochastisch unabhängig von den zuvor besuchten Zuständen und Aktionen sind.
Dass sie das künftige Verhalten des Systems beeinflussen, ist bei diesen Größen klar.
So werden Gebraucht- und Neupreis des Konkurrenten noch für die erste Hälfte dieses Schrittes direkt für das Kundenverhalten verwendet.
Der Rückkaufpreis beeinflusst das Verhalten der Eigentümer, deren Anzahl wiederum durch die Anzahl der Elemente in Zirkulation gegeben ist.
Der eigene Lagerstand erklärt durch Lagerkosten- und strafen einen Teil des eigenen Gewinnes, genau wie der Lagerstand des Konkurrenten einen Teil seines Verhaltens erklärt.

Der Schrittzähler als Teil des Zustands verdient noch einmal eine kurze Diskussion.
In zahlreichen Literaturbeispielen wird er nicht benötigt, hier ist er aber wegen der festen Episodenlänge ein Teil des Zustands.
So kann es nämlich einen Unterschied machen, ob man eine ansonsten identische Beobachtung am Anfang oder am Ende einer Episode macht.
Ein Beispiel dafür ist eine Situation mit einem relativ vollen Lager.
Am Anfang kann es sinnvoll sein, einen Teil der Elemente aus dem Lager zu einem niedrigen Preis abzuverkaufen, um langfristig Lagerkosten zu sparen, auch wenn das kurzfristig zu Verlusten führt.
Am Ende ist jedoch kein Raum mehr für langfristige Planung, und es werden hohe Verkaufspreise für kurzfristige Gewinne gesetzt.

Obwohl die Anzahl der zirkulierenden Elemente und der Lagerstand des Konkurrenten für die Markov-Eigenschaft notwendig sind, bereitet die Messung dieser beiden Werte Probleme.
Der Konkurrent lässt sich vermutlich nicht ins Lager schauen und die Anzahl der Elemente in Zirkulation kann allenfalls geschätzt werden.
Deshalb ist es eine zu beleuchtende Frage, ob die untersuchten Verfahren auch mit diesem \textit{partiell beobachtbaren Markov-Entscheidungsprozess} zufriedenstellende Ergebnisse liefern.

Jetzt muss noch einmal beleuchtet werden, dass für den beschriebenen Markt auch keine weiteren Größen beobachtet werden müssen und somit die Markov-Eigenschaft tatsächlich erfüllt ist.
Das Verhalten des Marktes ist vollständig definiert und kommt mit den genannten Größen aus.
Auch die regelbasierten Wettbewerber verwenden für ihr Verhalten nur die aufgezählten Variablen.
Messgrößen wie die vorher gesetzten Preise, die Anzahl der weggeworfenen Objekte oder die Anzahl der bisherigen Verkäufe könnte man im Zustand vermuten (sie sind ja schließlich auf der grafischen Visualisierung des Marktes), beeinflussen aber nicht das künftige Verhalten.
Eine Gedächtnisfunktion der Kunden (und dann notwendigerweise auch der Agenten) über vergangenes Verhalten ist eine mögliche Erweiterung, die das Modell praxisnäher bringen würde.
So könnten Kunden die vergangene Preispolitik in ihre Entscheidung mit einbeziehen und Anbietern ein höheres oder niedrigeres Renommee zubilligen.

\section{Regelbasierte Strategien -- Benchmark und Wettbewerber}
