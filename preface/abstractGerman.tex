Der Handel mit gebrauchten Artikeln auf Online-Marktplätzen ist ein wachsendes Geschäft.
Dabei stellt gerade das Pricing eine große Herausforderung dar.
Onlinehandel ist hochdynamisch, die Produktpalette ist groß, und es muss schnell auf Wettbewerber reagiert werden.
Daher ist die Automatisierung von Preisstrategien ein notwendiger Schritt für Händler.
Diese Arbeit befasst sich mit der dynamischen Preisgestaltung mit Reinforcement Learning (RL), einer Machine-Learning-Technologie.
Dank intensiver Forschung in den letzten Jahren gibt es heute viele verschiedene RL-Algorithmen.
In dieser Arbeit werden fünf verschiedene Algorithmen in Monopol-, Duopol- und Oligopolmärkten mit wettbewerbsorientierten regelbasierten Strategien verglichen.
Es zeigt sich, dass A2C, PPO und SAC erfolgreich sind und die Konkurrenten übertreffen können, während DDPG und TD3 nicht die nötige Leistung aufweisen.
Später werden mehrere Probleme, die in realen Anwendungen auftreten, angegangen und gelöst.
Es wird gezeigt, dass das Fehlen einiger Informationen über den Konkurrenten die Algorithmen nicht beeinträchtigt.
Das Problem, die Konkurrenzstrategie nicht zu kennen, kann mit Self-Play erfolgreich gelöst werden.