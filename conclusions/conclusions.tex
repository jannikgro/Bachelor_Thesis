Obwohl ein Recommerce-Markt zahlreiche Herausforderungen mit sich bringt und die Konkurrenz stark ist, konnten sehr effektive Preisstrategien erfolgreich mit Reinforcement-Learning-Verfahren gefunden werden.
Dabei lieferten die Algorithmen A2C, PPO und SAC sehr gute Spitzenergebnisse, während DDPG und TD3 nicht gut abschnitten.
Advantage Actor Critic ist den beiden Verfahren PPO und SAC in der Trainingsstabilität deutlich unterlegen, konnte aber sehr gute Policies liefern, wenn die Modellparameter während des Trainings in einem Moment guter Leistung gespeichert wurden.
Weiterhin übertraf Advantage Actor Critic die anderen Algorithmen bei der Trainingsdauer und überraschenderweise auch bei der Sample Efficiency.
Im Duopol und Monopol erreichte PPO die beste Performance der verglichenen Algorithmen.
Über unterschiedliche Szenarien hinweg schnitt er in der Spitzenleistung am besten ab und lieferte stets stabiles Training.
Sein Nachteil war in jeder der Analysen die geringere Sample Efficiency.
Der Soft Actor Critic Algorithmus lieferte ebenfalls äußerst stabile Trainingsläufe, blieb jedoch in seiner Spitzenleistung auf Duopolszenarien hinter PPO zurück.
In der Sample Efficiency hingegen war SAC PPO stets überlegen.
Im deutlich höherdimensionalen Oligopol-Markt lieferte SAC die beste Spitzenleistung und war PPO überlegen.
Eine genauere Untersuchung dieses Phänomens legt nahe, dass PPO weniger gut komplexe Policyfunktionen trainiert, die durch größere neuronale Netze berechnet werden.
Weil die Algorithmen in den unterschiedlichen Szenarien und nach unterschiedlichen Bewertungskriterien unterschiedlich gut abschnitten, kann kein klarer Sieger gekürt werden.

Die Ergebnisse dieser Arbeit sind eine Bestätigung, dass Reinforcement Learning auf realen Märkten zur Optimierung des Pricing angewandt werden kann.
Der verwendete Markt ist kompliziert genug und die Ergebnisse stabil genug, sodass erwartet werden kann, dass die Algorithmen auch auf Märkten mit weiteren Parametern wie Qualität oder Saison erfolgreich abschneiden werden.
Weiterhin wurde die Lösbarkeit zentraler Probleme für die praktische Anwendung untersucht.
Die Ergebnisse fielen positiv aus.
So funktionieren die Algorithmen auch dann, wenn einige Aspekte des Marktes nicht beobachtet werden können und die Markoveigenschaft damit nicht erfüllt ist.
Das Problem unbekannter Konkurrenzpolicy kann durch Self-Play gelöst werden.
Die dabei trainierten Policies konnten ihre Konkurrenz bei guten Ergebnissen übertreffen, ohne sie jemals im Training gesehen zu haben.

Als Herausforderung verbleibt das nicht genau bekannte Verhalten von Käufern und Eigentümern für das Training auf Simulationsplattformen.
Training auf Simulationsplattformen ist aber notwendig, weil wegen des weiterhin hohen Samplebedarfes und schlechter Ergebnisse während des Trainings nicht direkt auf dem echten Markt trainiert werden kann.
Um die Simulationsplattform auf den realen Markt anzupassen, muss das Verhalten dieser Marktteilnehmer aus Mangel an akkuraten theoretischen Modellen aus Echtweltdaten geschätzt werden.
Das ist jedoch mit verschiedenen Regressionsverfahren möglich und wurde in vorheriger Forschungsarbeit bereits behandelt, zum Beispiel beim zweistufigen Verfahren in \cite{10.1145/3219819.3219833}.

Eine denkbare Alternative zum Training mit einer Simulationsplattform ist es, auf einem Markt die Daten, die mit einer bisherigen Preisstrategie gesammelt wurden, direkt in einen Experiencebuffer zu schreiben.
Mit einem Off-Policy-Verfahren wie SAC könnte damit eine Policy erlernt werden, die auf dem echten Marktplatz bereits akzeptable Ergebnisse liefert.
Anschließend könnte auf dem echten Markt noch Fine-Tuning-Training stattfinden.
Stabilität und Sample-Efficiency sind die Haupteigenschaften, die ein genutztes RL-Verfahren mitbringen müsste.
Ob ein solches Vorgehen praxistauglich ist, sollte Gegenstand künftiger Forschung sein.

Eine weitere Schwierigkeit, die auf realen Marktplätzen auftreten kann und nicht im Umfang dieser Arbeit war, ist ein Gedächtnis der Kunden und Eigentümer.
In diesem Modell kommen Kunden und Eigentümer auf den Marktplatz und haben für ihre Entscheidung nur die Informationen eines einzigen Zeitschrittes.
Tatsächlich ist aber davon auszugehen, dass zumindest einige Kunden den Preisverlauf beobachten und nicht nur in Abhängigkeit aktueller, sondern auch vergangener oder antizipierter zukünftiger Preise ihre Kaufentscheidung treffen.
Stünde dem Agenten dann wie im derzeitigen Modell nur der aktuelle Markt zur Verfügung, so fehlte ihm ein großer Teil der eigentlich benötigten Entscheidungsgrundlage.
Um eine längere Liste vergangener Zustände zu verarbeiten, ohne dass von Menschen relevante Features bei der Entwicklung extrahiert wurden, werden andere Arten von neuronalen Netzen benötigt.
Wenn diese Informationen eine feste Größe haben, könnte über den Einsatz von CNNs nachgedacht werden, bei variabler Größe fiele die Wahl auf rekurrente Netze oder Transformer-Modelle.